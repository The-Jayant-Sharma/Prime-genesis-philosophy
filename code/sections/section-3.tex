\section{Conclusion}

In this paper, we have explored and proposed a philosophical principle that suggests creation unfolds according to the sequence of prime numbers. At the heart of this view lies the following proposition:

\begin{center}
\emph{
"Since unity is the most fundamental element, there must have been only one in the beginning.\\
Creation proceeds from that One — a reality even beyond existence and inexistence.\\
From this source, a division arises: into existence (analogous to infinity — maximal potential),\\
and inexistence (analogous to zero — absence of potential).\\
Existence, having infinite generative capacity, builds upon inexistence to give rise to all things.\\
Yet the process of creation unfolds not arbitrarily, but in alignment with the sequence of prime numbers:\\
$1, 2, 3, 5, 7, \dots$ — which we have proposed as the central rhythm of genesis.\\
What occurs after the creation of this first pair, however, remains fundamentally unknowable,\\
for zero and infinity together possess the power to create anything."
}
\end{center}

Another major principle we explored and proposed in the paper is,

\begin{center}
\emph{"The creation should unfold as unfolds the sequence of primes"}
\end{center}

This concludes our consideration.

\section*{References}
\begin{itemize}
  \item G. H. Hardy and E. M. Wright, \textit{An Introduction to the Theory of Numbers}, 6th ed., Oxford University Press, 2008.
  \item Plato, \textit{Timaeus}, trans. Donald J. Zeyl, Hackett Publishing, 2000.
  \item Plotinus, \textit{The Enneads}, trans. Stephen MacKenna, Penguin Classics, 1991.
\end{itemize}

