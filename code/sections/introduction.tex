\section*{Word}
\begin{center}
\emph{"If logic is vision, then philosophy is faith. The one who possesses both, becomes a seer. But the one who mistakens either of them for the other, becomes an ignorant. \\
The truth should not be accessible to faith only, but to an open mind. If anything deludes from equality and unity then it is certainly an ignorance. If anything deludes into just ideas, it is certainly an ignorance. \\
Keeping an open mind entirely, even for vision and faith is what makes one a seer."}
\end{center}

\section{Introduction}

Across diverse traditions, every process of creation appears to follow a structure that resonates with the sequence of prime numbers.

In Eastern philosophies, creation begins with the One — the undivided source. From this One emerged desire, and thus the One split into Two. This duality marked the first distinction within the divine, initiating the cosmos. Soon after appeared the Three modes of action (*gunas*), followed by the emergence of Five senses, and then the Seven chakras — energy centers of the body and soul. Each step corresponds not only to a metaphysical unfolding, but also aligns with the progression of prime numbers: $1, 2, 3, 5, 7$.

A similar pattern can be discerned in Western traditions. In the beginning stood the Unknown God — unmanifest and singular. Then came the division into Two — Creator and Creation. Thereafter appeared the Holy Trinity — Father, Son, and Spirit — forming a divine Three. Subsequently, the number Five gains prominence: the five books of the Torah, five fingers of the hand, five senses of perception. Then comes Seven — the number of spiritual completeness — expressed through the seven days of creation, and echoed in the Book of Revelation with seven seals, seven trumpets, and seven churches.

This pattern seems more than symbolic — it suggests a foundational law of emergence.

Even in mathematical logic, an analogous structure can be imagined. First, we posit something that is common to both zero and infinity — a transcendent concept we shall call *Iritrium*. This Iritrium lies beyond all mathematical comprehension — the source of both existence and inexistence. It splits into **zero** and **infinity** — the poles of emptiness and boundlessness. From these arise the rest of mathematical objects. Just as infinite zero-dimensional points form a line, and infinitely many one-dimensional lines form a plane, the universe builds itself layer by layer from this prime metaphysical split.

A similar sequence appears in the genesis of music. Before sound or silence existed, there was *Iritrium* — beyond even the being or non-being of music. From this emerged the duality of **existence** and **inexistence** of sound. Following that came the three primal vowels or sonic roots, which are recognizable in the Devanagari script as the foundational phonemes: \textbf{a}, \textbf{i}, and \textbf{u}. Possibly, we may then descend into a structure that echoes the number Five — perhaps in the elemental tones or foundational resonances of sound.

All these sequences — from metaphysics, mathematics, and music — do not appear coincidental. Rather, they point toward a unified law: a *Prime Genesis Principle*. This principle proposes that **creation speaks in the language of prime numbers** — that the very unfolding of existence mirrors the rhythm of irreducible quantities.

Hence, we propose the following foundational insight:

\begin{principle}[Principle of Prime Genesis]
The genesis of any system unfolds in resonance with the sequence of prime numbers.
\end{principle}

