\section{Philosophy}

If something is to unfold, it must do so from the most fundamental layers of reality.

In mathematics, the sequence of **prime numbers** is arguably the most fundamental increasing sequence. By definition, a prime number is a number that is **not divisible by any other number except one and itself**. This definition reflects a **multiplicative restriction**: such numbers cannot be factored through multiplication by any other non-trivial component. Notably, **division is the inverse of multiplication**, so the definition of primes encodes a kind of **multiplicative irreducibility**.

Now, if we probe even deeper, we may ask: is there an **additive analogue** to this idea? That is, can we define a set of numbers that are **not decomposable into a sum of any two numbers other than the additive identity and themselves**? In this case, the only number that satisfies this condition is **unity ($1$)**. It cannot be expressed as the sum of two positive integers — making it **additively irreducible**.

This observation reinforces the foundational role of **unity** in the architecture of creation. Even the most fundamental sequence in mathematics — the primes — is, in a sense, **composed from unity**, as all primes are sums of multiple $1$s. Furthermore, addition is the **most basic form of numerical interaction**. Even the notion of “stepping” from one number to the next involves addition — it is how numbers move and evolve.

However, the **additive irreducible sequence** (which contains only the number $1$) is not an increasing sequence. And for any creative process, **change** is essential. Repetition of unity alone does not generate complexity. Hence, we regard the **primes** — the next most fundamental structure — as the **creative foundation**, because they represent **change** while preserving a kind of **irreducibility**.

Yet, prime numbers do **not include unity**. So why do we still observe the number $1$ appearing at the very beginning of so many creation systems?

Because creation must begin from the **absolute fundamental**, and unity is the **only element** of the additive irreducible sequence. Thus, even though the prime sequence governs the **unfolding**, the **origin itself** is unity. This may explain why **unity always appears at the beginning** — as the spark from which even the prime rhythm of creation begins.
This explains why creation unfolds through the sequential order of 1, 2, 3, 5, 7, and potentially beyond.

\subsection{The Role of Addition and Grouping}

In the very beginning, there exists only the number \emph{one}. By adding one to itself, we obtain \emph{two}. Adding one again gives \emph{three}. However, to reach \emph{five}, we must now add \emph{two}, and adding two again gives \emph{seven}. Notably, we cannot simply add two to seven to reach the next prime, indicating a shift in the pattern.

Here, \emph{addition} represents more than a mathematical operation—it suggests \emph{the type and amount of energy required to move to the next stage}.

Based on this, we divide the early prime numbers into three philosophical groupings:

\paragraph{The divine number}  
This group includes only the number \emph{one}. It is the most fundamental quantity in both mathematics and philosophy. It represents the source of creation—indivisible, self-sufficient, and foundational to all that follows.

\paragraph{The heavenly numbers}  
This group includes the numbers \emph{two} and \emph{three}, both formed by adding unity.

The number \emph{two} comes from adding one to one. It carries the properties of unity and was born directly from it. The number \emph{three} follows the same logic, being the result of adding one to two. Both these numbers emerge from the influence of unity and belong to the same energetic pattern.

Philosophically, \emph{two} may represent the first division in origin, such as the separation between \emph{existence} and \emph{inexistence}. It can also symbolize \emph{desire}, which arises when there is a second—something to reach for or long for. In this way, \emph{three} becomes the first product of desire and duality, echoing how birth requires two and how the process of birth itself reflects a kind of sacred longing.

\paragraph{The holy numbers}  
This group includes the numbers \emph{five} and \emph{seven}. These numbers are no longer produced by the energy of unity, but by the energy of duality. For example, five is formed by adding two to three, and seven is formed by adding two to five.

These numbers belong to the world that comes \emph{after} desire. They mark stages in the construction of more complex structures. The number \emph{seven} in particular has often been associated with spiritual completion or checkpoints in many traditions. It appears as the number of days in creation, the number of energy centers in the body, and other sacred systems. After seven, the unfolding tends to become more material in nature.

This classification reflects how each stage of creation may require a different kind of energy. In this view, the early primes serve not just as numbers, but as signs of the pattern through which creation steps forward.

\subsection{The Origin: Iritrium}

If there must be only one at the beginning, then it must lie beyond all forms of duality. In particular, it must transcend the very first pair to emerge: \emph{existence} and \emph{inexistence}. 

Thus, the true origin must be something prior to — and beyond — both existence and inexistence. It is not something that can be fully captured through ideas or philosophies. At best, it can only be hinted at. We refer to this pre-dual source as \emph{Iritrium}. However, it is essential to recognize that Iritrium is beyond all definition. It cannot be described by properties, for properties themselves belong to the domain of existence. And since Iritrium is beyond both existence and inexistence, it necessarily escapes all conceptual frameworks.

From this unknowable source, Iritrium, the first distinction arises: the emergence of existence and inexistence. This moment marks the beginning of creation.

\subsection{The First Pair: Existence and Inexistence}

As discussed above, Iritrium precedes and transcends the categories of existence and inexistence. Once these two arise, they must together give rise to all further creation. In other words, the combination of existence and inexistence must contain the potential to produce everything — all that can ever be.

The only pair in mathematics that mirrors this generative duality is the pair of \emph{zero} and \emph{infinity}. Consider their creative potential: an infinite collection of zero-dimensional points can form a line of any length. An infinite number of zero-dimensional lines can form a two-dimensional surface. The structure of space itself seems to emerge from this tension. Even in analysis, the expression $0 \times \infty$ is \emph{indeterminate}, symbolizing the unknown creative outcome of their interaction.

No other pair of values seems to possess this fundamental power — the ability to bring forth form from non-being.

Hence, in this cosmological analogy, \emph{inexistence} corresponds to the nature of zero, and \emph{existence} to the nature of infinity. These two — rooted in the Iritrium — form the first generative polarity of creation.

\subsection{Beyond the First Pair}

Since the interaction between zero and infinity can result in any form or structure, it is not determined by them alone what should arise first. Whether everything is created simultaneously, sequentially, or selectively remains inherently unknowable.

Thus, the specific unfolding of creation beyond the first pair — its direction, form, and rhythm — remains fundamentally concealed. What follows from the Iritrium through the duality of existence and inexistence is veiled in mystery, suggesting that the deeper logic of creation is not fully accessible, but only observable through its patterns.

